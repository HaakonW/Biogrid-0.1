\documentclass[12pt, oneside]{article}
\usepackage[utf8]{inputenc}   			%Active æøå
\usepackage{geometry}				% See geometry.pdf to learn the layout options. There are lots.
\usepackage{color}  
\geometry{a4paper}                   		
\usepackage[parfill]{parskip}    		
\usepackage{graphicx}					
\usepackage{amssymb}
\usepackage{hyperref}
\hypersetup{
	colorlinks=true,
	linkcolor=black,
	urlcolor=red,
}

\renewcommand{\contentsname}{Innholdsfortegnelse} 

\title{Prosjektdagbok - Biogrid}
\author{Peter Wilhelmsen}
%\date{}							% Activate to display a given date or no date

\begin{document}
\maketitle
\pagebreak
\tableofcontents
\pagebreak

\section{Introduksjon}
Våren 2016 gjennomfører gruppe 20 bachelorprosjekt på Høgskolen i Oslo og Akerhus i samarbeid med Biogrid Cortex AS. \\
Dagboken er delt opp i måneder, som igjen er delt opp i underpunkter. 

% <---------------------------- Start dagbok -------------------------------->
\section{Dagbok}


% <----------------------------- November 2015 ----------------------------->
\subsection{November 2015}
Tok kontakt med Biogrid på mail og Oscar er i møte med August Flatby som er daglig leder for Biogrid AS.
En uke senere er vi igjen i kontakt med Biogrid og de ønsker å samarbeide med oss. Gir oss tid til å tenke på tilbudet, men vi har ingen med å takke ja til tilbudet. Avtaler å ha møte 14.desember i Biogrids lokaler når Peter er tilbake i Oslo.


% <---------------------------- Desember 2015 ----------------------------->
\subsection{Desember 2015}

\subsubsection{ Møte med Biogrid 14.12.15}
Oscar og Peter i møte med August Flatby. Får presentert prosjektet og ideer Flatby har. Får innføring om diverse teknologier som Biogrid kan tenke seg skal brukes. Disse er:  
\begin{itemize}
	\item Rickshaw.js for grafisk fremstilling
	\item Bootstrap
	\item Meteor
	\item MongoDB
\end{itemize}
Biogrid gir beskjed om at vi står fritt til å komme med forslag selv til hva som kan brukes. Avtaler å møtes igjen over nyttår.

Tiden frem mot nyttår går med til å se litt på aktuelle pakker i JavaScript platformen Meteor. 


% <--------------------------- Januar 2016 -------------------------------------->
\subsection{Januar 2016}

\subsubsection{Forprosjektrapport 07.01.16}
Møtes hos Haakon der vi diskuterer litt og går gjennom hva som skal være med i en forprosjektrapport.\\
Sender mail til Biogrid der vi ber om møte. Får svar dagen etter. Han er opptatt med å lage til oppsett på hvordan dataen skal lagres, ingen respons på forespørsel om møte.



\subsubsection{12.01.16}
Gruppen møtes på skolen. Fortsatt ikke fått avtalt møte med Biogrid. \\
Gruppen diskuterer bruken av GitHub og hvilken editor vi skal bruke til rapportskriving. Alternativene er Word Online, Pages og Google Docs. Gruppen blir enig om at Word er det beste alternativ da dette er editoren som har mest funksjonalitet tilgjengelig. 



\subsubsection{14.01.16}
Gruppen møtes hos Haakon. Sender ny mail til Biogrid for å avtale møte. \\
Oppretter dokument for forprosjektrapporten, som foreløpig kun har vært notater. Gruppen kommer et godt stykke med rapporten, venter kun på Biogrid for å avtale endelige betingelser, mål og lignende.



\subsubsection{Videomøte med Biogrid 15.01.16}
Har videomøte med Biogrid på \url{http://www.appear.in}. Tar opp noen punkter vedrørende forprosjektrapporten. Hele gruppen deltar. Har på forhånd sendt utkastet av rapporten til Biogrid. Det er gjort noen små endringer, men ellers fornøyd med hvordan den er strukturert. \\

Oppsummering av hovedpunkter:

Løsning består av:
\begin{itemize}
	\item IBM Softlayer
	\item Docker: Løsningen deployes i Docker
	\item Node-RED: data stream processor
	\item Mosquitto, MQTT Broker
	\item MongoDB
	\item Meteor: selve applikasjonen\\
	Pakkebruk i Meteor. Mange fine løsninger, men redd det kan bli en for stor snarvei. Samtidig er det mye kode å sette seg inn i og finne ut hvordan 	fungerer. Tar dette opp veileder. 
\end{itemize}
Vet ikke hvor nøye det er hvor mye vi forklarer hvert enkelt hovedpunkt i løsningen siden vi konsentrerer oss om Meteor og MongoDB. Skissen vi fikk tilsendt passer fint inn her, men da kanskje greit å lage en kort forklaring til den.

Det skal skrives kontrakt mellom skolen og oppdragsgiver. Gruppen skal kontakte veileder for møte. Foreløpig ingen informasjon hvordan kontrakten fungerer. Gir beskjed til Biogrid etter møte med veileder. 


\subsubsection{Møte med veileder Thor Hasle 21.01.16}
Oscar og Peter i møte med gruppen veileder, Thor Hasle. Presenterer prosjektet og får innspill fra Thor.
\begin{itemize}
	\item  Ingen faste møter med Hasle. Ta kontakt når vi trenger, kan bruke han som konsulent.
	\item  Kontrakt mellom arbeidsgiver og HiOA: bruke standardkontrakt som finnes på nettside for bachelor.
	\item  Begynne på kravspesifikasjon og arbeidsplan vil være neste steg.
	\item  Tips: definere Internet of Things i rapporten.
\end{itemize}

\subsubsection{22.01.16}
Gruppen og Biogrid bruker Slack som kommunikasjonsmiddel. Dette er en applikasjon som gjør det lett å sende meldinger, dele filer og lignende.

August sender over linker om ReactJS. Det er JS bibliotek som brukes til å lage UI-komponenter. Det er spådd at React er up and coming, og vi får to artikler å lese. React virker bra å bruke og gruppen starter å se på tutorials og sette seg inn i biblioteket. 

Sender over kontrakten mellom skolen og Biogrid, og vi avtaler møte 4.februar når de er på plass i nye lokaler på Startup Lab i Forskningsparken.

\subsubsection{Kravspesifikasjon 25.01.16 - 03.02.16}
Møtes på skolen for å starte på kravspesifikasjoen. Diskuterer hvilke funksjonelle og ikke funksjonelle krav, milepælsplan. Vanskelig å sette noe veldig konkret da det fortsatt trengs innspill fra Biogrid om ønsket funksjonalitet osv.\\
Gruppen må få innføring i hvordan Docker og IBM Softlayer funker. Gruppen ønsker også en kjapp innføring i bruken av Github da den ikke føler seg helt komfortabel med bruken av versjonshåndteringssystemet. 


% <-------------------------------  Februar 2016 ----------------------------->
\subsection{Februar}



\subsubsection{Møte med Biogrid på Startup Lab - 04.02.16 }
Hovedpunker i møte:
\begin{itemize}
	\item Softwaren har gått fra å være tiltenkt kun innen landbruk til å kunne brukes i flere hold. Skal utvikles generisk slik at det kan passe flere 			formål. Det finnes lignende løsninger, men vi skal ikke skremmes av at vår løsning vil ligne/gjøre det samme som dem. Denne løsningen 			skal ikke stagnere, men kunne utvikles videre.
	\item Hvem skal bruke systemet?\\
	 	Målgruppen er fortsatt ikke helt sikker. Det er IKKE ment for hobbybruk, men for en industri som for tiden ligger an til å være 					innen landbruk, jordbruk eller lignende. 1. mars skal Biogrid ha seminar for bedrifter innen landbruk.
	\item Arkitekturen\\
		Docker er som en virtuell maskin, men kalles en container. Det er én container pr prosess der hver container tror den er en maskin. 				Containerne har hvert sitt filsystem og hersker over seg selv. Containerne kan også dele filsystem. En container kan tas ned uten å påvirke 		de andre. Containere kan snakke med hverandre. Docker Compose brukes til å starte containerne samtidig.(script). Docker lokalt er helt likt 		som på en server, veldig lett å sette ut i virkeligheten uten modifikasjoner.\\
	
		 BiogridCortex består av: - Cortex: orkestrer alt sammen. \\
		 Mongo\\
		 Mosquitto\\
		 NODE-RED\\
		 Meteor. (foreløpig et lite spørsmåltegn til bruken av plattformen).\\
		 MQTT BROKER Hvis noen er interessert i disse dataene sendes de videre. Har en slags adressering. Du sier ikke hvem du er, men hvor 		du vil sende dataene. TOPIC: /sensor/<type> /sensor/co2 payload: ID:GUID (Global Uniqe ID) VALUE: 413.6
		 
	\item Milpæler
		\begin{enumerate}
			\item Visualisere data
			\item Liste av sensorer
			\item Logge inn - få mulighet til å velge hvilken lokasjon du vil se data for
		\end{enumerate}
		
	\item Fremdrift videre
		\begin{description}
			\item Sette opp arbeidsmiljø
			\item Velge editor. Biogrid mener Atom er et godt alternativ.
			\item GitHub. Bruk Sourcetree, mange mener GitHub Desktop er for simpelt.
			\item Sjekke forskjellige bibliotek for grafer. Sammenlign og finn beste alternativ.
			\item UX. Biogrid mener Bootstrap er et godt alternativ. Gruppen har brukt det litt før og er enig. 
			\item Implementasjon
			\item Deretter begynne å gruppere sensorer og bygg/lokasjoner
			\item User account. Vil ikke bruke tid på å logge inn og ut under utvikling
		\end{description}
\end{itemize}

Oppsummering:\\
Fikk satt opp arbeidsmiljø(Docker) og August viste oss hvordan dette funket. Hele gruppen er online og kan starte generering av data til MongoDB. Fremover vil vi lese oss opp på forskjellige alternativer med tanke på fremstilling av data: fokus på interaktivitet, estetikk, må kunne oppdatere data i realtime og skal ikke være kommersielt.

Det må leses opp om Docker og MongoDB.

Største vending/overraskelse: Biogrid har begynt å tvile på Meteor som plattform. Meteor bygger sammen alt for deg, som gir mindre kontroll. Vil at gruppen sjekker opp andre webteknologier for å se om det finnes andre gode alternativer. Tenk også på hva gruppen har lyst og er interessert i å lære. Meteor er IKKE utelukket. 



\subsubsection{JavaScript bibliotek for grafer 05.02.16}
Gruppen møtes for å finne ut av hvilke biblioteker for fremvisning av grafer som er aktuelle. Gruppen har lest artikler om anbefalte biblioteker:\\
\url{http://thenextweb.com/dd/2015/06/12/20-best-javascript-chart-libraries/}\\
*FLERE LINKER*\\
Å kunne zoome er en viktig funksjon gruppen ser etter. Det må være mulig å kunne zoome seg inn fra dag, til time og til minutt. De tre aktuelle bibliotekene er:
\begin{itemize}
	\item Dygraphs: et open source JavaScript bibliotek som lager grafer. Det er interaktivt slik at det er mulig å forandre grafen ved å zoome inn på 		ønsket periode. På papiret ser det veldig aktuelt ut. Rent designmessig ser det ikke så bra ut, men vi regner med at dette er noe som kan 		forandres på senere. \url{http://www.dygraphs.com}
	\item Flot: et JavaScript bibliotek for jQuery. Også interaktiv som gjør det mulig å zoome inn på grafen for å få et mer detaljert bilde. Brukt av 			veldig mange forskjellige bedrifter. God dokumentasjon, med video blant annet. Ser kanskje litt bedre ut en dygraphs. Siden det bygger på 		jQuery er de fleste browsere også støttet. Støtter JSON. Virker enkelt å lære, men kanskje det blir for enkelt for oss. \url{http://www.flotcharts.org}
	\item Rickshaw: Javascript rammeverk som er open source. Ble anbefalt dette av arbeidsgiver første gang, men med forbehold om at det kan 			finnes bedre alternativer der ute. Muligheter for interaktive grafer her også. Støtte for charts med time-series data. Reklamerer med at de kan 		lage grafer basert real time oppdatering. Bygger på D3 som skal være et godt rammeverk. \url{http://code.shutterstock.com/rickshaw}
\end{itemize}



\subsubsection{Evaluering og valg 09.02.16}
Frem mot i dag har tiden gått med på å lese om React, finne ut om Meteor er godt nok til å være gruppens webplattform og sjekke de grafiske JS bibliotekene og lese om Docker. 
\begin{itemize}
	\item Etter møte forrige uke var August litt betenkt på bruken av Meteor, men i dag virker det som har tror det vil funke bra. Han sender oss en 	link på bok/pdf om 			Meteor som han mener kan være en god ressurs. Dette tar vi som et tegn på at han syns gruppen skal gå for Meteor! Peter har funnet artikkel (\url{http://t			utorialzine.com/2015/12/the-languages-and-frameworks-you-should-learn-in-2016/}) om at Meteor vil være en platform som er "up and coming" i 2016. Leter 			fortsatt etter flere artikler som understøtter denne påstanden. Sier at Meteor er en hybrid mobile framework og at det er verdt å lære.
	\item Valg om editor lander på Atom for alle tre. Atom har flere tilleggspakker man kan installere for å optimalisere og tilpasse bruk:\\
		 - Language støtte for ReactJS: jsx.\\
		 - Bootstrap autocomplete. Kan være denne ikke vil fungere sammen med ReactJS pakken pga class ikke brukes i html-tagger i react, det 		er erstattet med className.
	\item August har gjort endringer på hvordan data legges inn i collection i MongoDB. I stedet for at det skal være en collection for hver sensor 			type, er det nå 1 collection som heter sensor, der hvert dokument har  sensor-id, day og type: rh, co2. Endringene er pushet til 			GitHub og funker for gruppen.
\end{itemize}



\subsubsection{Meteor i kombinasjon med ReactJS 15.02.16}
\begin{itemize}
	\item Biogrid har sett på alternative og foreslår å bruke Dygraphs. Dette var den gruppen likte best. Har flere funksjoner, men ser ikke veldig pent 		ut. Har sett på 		docs om CSS styling, og det skal være mulig å tilpasse slik gruppen måtte ønske.
	\item Biogrid har sett mer på webstacken og selve React. Sier React virker veldig bra. Sender link på video som forklarer hvorfor Facebook 			utviklet react: 		\url{https://facebook.github.io/flux/docs/overview.html} (se fra 24.00 ca).
	
		Viktig: React brukes kun for å lage UI-komponenter.
	\item Biogrid mener Meteor er en god måte å holde data syncet mellom server og klient, så det vil satses på en Meteor-React kombinasjon. 
	\item Gruppen har prøvd seg på tutorialer om React:\\
		\url{https://www.youtube.com/watch?v=kVbVBp35keQ}. Denne tutorial bruker Flowrouter og react-layout, to pakker fra Kadira som er "perfomance monitoring for 			meteor" - \url{http://www.kadira.io}. Dette virker som to gode pakker for å håndtere routing og hvordan man bygger opp layouter i React. Gruppen sjekker fortsatt alternativer 		til routing og hvordan man kan bruke react. Slik det ser ut finnes det flere fremgangsmåter.
\end{itemize}



\subsubsection{Fra database til graf 18.02.16}
\begin{itemize}
	\item Haakon og Oscar har startet med å hente data fra MongoDB og sette disse inn i graf ved hjelp av Dygraph. Pushet TestKode til GitHub. Peter får kjørt 			Meteorprosjektet på sin maskin. Planen er at Oscar og Haakon fortsetter på dette i morgen.
	\item Peter er hjemme i Ålesund. (11.02 - 25.02(?)). Har lest gjennom AirBnb JS styleguide, \url{https://github.com/airbnb/javascript}, og leser om React. Skal lage oversikt over pakker og alternativer på hvordan 		React skal brukes.
	\item Haakon funnet Robomongo. Applikasjon som man kan koble opp mot MongoDB. Oversiktlig GUI, der man lett kan se de forskjellige documents og gjøre 				spørringer. Slepper å gjøre spørringer i terminal.
	\item Haakon har gjort oppdatert hjemmesiden til prosjektet. Lastet opp forprosjektrapport. 
\end{itemize}



\subsubsection{Latex og graf 19.02.16}
Peter har testet ut Latex som teksteditor for rapportskriving. Latex gjør det lett for gruppen å jobbe på samme dokument, og man trenger ikke å logge seg på Word Online. Latex gir ryddig dokumentstruktur og det er lett å generere overskrifter, bilder/figurer og innholdsfortegnelse. 

Haakon og Oscar jobber videre med spørring mot databasen. 

Problemstillinger: Hvilken form for routing skal vi bruke? Peter har lest opp om Flow-router og React-router, begge ser ut til å være gode alternativer. Det finnes mer dokumentasjon på React-router, men Flowrouter har blitt brukt av mange i kombinasjon med Meteor. Lest denne linken: \url{https://forums.meteor.com/t/react-router-vs-flow-router/6867}

Link til ReactJS Github med React-router \url{https://github.com/reactjs/react-router}.



\subsubsection{Spørringer mot Mongo 20.02.16}
I dag jobber gruppen med spørringer mot databasen. Utfordringen er å få til en spørring som finner første sekund for hvert minutt for hver time(?).
Er det mest effektivt å gjøre én spørring og lagre resultatet som et objekt for å kunne jobbe på dette "lokalt"?



% <-------------------------- MARS ------------------------------------------> %
\subsection{Mars}


\subsubsection{IOT-seminar på Startup Lab 01.03.16}
I dag arrangerte Biogrid et seminar om Internet of Things på Startup Lab i forskningsparken. Her var flere aktører involvert, der i blant Tine, Innovasjon Norge, Trådløse Trondheim, Norsvin og Telenor. Fullstendig liste skal bli mailet til oss. Temaet for seminaret var hvordan landbruket kan ta i bruk Internet of Things: Hvordan bruke sensorer til å plukke opp data, sende dette til skyen, for så å kunne bruke det til analyse og hente ut verdier som kan være viktig for f.eks produsentene og rådgivere. 

August hos biogrid fortalte om sin tur til Singapore der han tok en nærmere kikk på Vertical Farming. Utfordringen til firmaet i Singapore var hvordan de kunne utvide til å overvåke planter på flere hustak uten å måtte flytte seg mellom alle takene. Dette har ført til tanken bak BiogridCortex: ha sensorer ute der du ønsker, send data til skyen og analyser og bruk dataene på ønsket måte.

Thomas Ulleberg fra Trådløse Trondheim/Wireless Trondheim fortalte om radiobølge-teknologi og LoRa. \url{https://www.lora-alliance.org/What-Is-LoRa/Technology}. 

7Sense var siste selskap ut. Et selskap som utvikler samme produkt som BiogridCortex og har dette i produksjon. \url{http://www.agrimon.no}



\subsubsection{Planlegging 02.03.16} 
Dagsplan:
\begin{itemize}
\item Fremdriftsplan og kravspesifikasjon. 
\item Arbeidsmetode: scrum/kanban\\
\url{https://www.atlassian.com/agile/kanban}\\
\url{http://www.agileweboperations.com/scrum-vs-kanban}
\item Forslag til design/layout. Ved å ha en grov design er det lettere å se for seg hva som skal gjøres på neste punkt. 
\item Finne ut hva som skal lages av komponenter,grafer og lignende. Prioritere oppgaver.
\item Fordele oppgaver for periode med utvikling. Sette opp ansvarlig for oppgavene.
\item Dokumentasjon: Hva kan skrives på nå? IoT, arbeidsmetoder, forklare teknologivalg og om de forskjellige teknologiene. Språket i dokumentasjonen må være konsistent: Unngå bruk av ord som "såpass, kjempe vanskelig" (type muntlig språk). Omtale oss tre som "Gruppen"? I de situasjoner det trenges ved navn. August representert som Biogrid siden han er vår eneste kontaktperson? Har gjort det på denne måten i store deler av dagboken. \\ 
Vær nøye på å huske/lagre referanser!
\item Få kontakt med Biogrid og svar på melding sendt på Slack 25.02.
\end{itemize}



\subsubsection{Sette opp layout og spørringer mot database 03.03.16}



\subsubsection{Mongo til minimongo, publish/subscribe 04.03.16}
Minimongo kan by på problemer med tanke på store mengder timeseriesdata, les mer her: \url{https://www.meteor.com/mini-databases}

\input{testfil.tex}


\end{document} 